\documentclass[10pt]{beamer}
\usepackage{kotex}

\usepackage{framed}
\usepackage{graphicx}
%https://www.overleaf.com/learn/latex/Inserting_Images

\usepackage{amsmath}
%use dfrac
\usepackage{xcolor}

\usepackage{amsthm}
%\usepackage{tabl}
\usepackage{listings}
\definecolor{mGreen}{rgb}{0,0.6,0}
\definecolor{mGray}{rgb}{0.5,0.5,0.5}
\definecolor{mPurple}{rgb}{0.58,0,0.82}
\definecolor{backgroundColour}{rgb}{0.95,0.95,0.92}
%https://tex.stackexchange.com/questions/348651/c-code-to-add-in-the-document
\lstdefinestyle{CStyle}{
    backgroundcolor=\color{backgroundColour},   
    commentstyle=\color{mGreen},
    keywordstyle=\color{magenta},
    numberstyle=\tiny\color{mGray},
    stringstyle=\color{mPurple},
    basicstyle=\footnotesize,
    breakatwhitespace=false,         
    breaklines=true,                 
    captionpos=b,                    
    keepspaces=true,                 
    numbers=left,                    
    numbersep=5pt,                  
    showspaces=false,                
    showstringspaces=false,
    showtabs=false,                  
    tabsize=2,
    language=C
}

\usepackage{url}

\usepackage{etoolbox}
\AtBeginEnvironment{quote}{\singlespacing\small}


\usepackage{thmtools}
\usepackage{xcolor}
\declaretheoremstyle[% spaceabove=6pt,spacebelow=6pt, headfont=\color{MainColorOne}\sffamily\bfseries, notefont=\mdseries, notebraces={[}{]}, bodyfont=\normalfont,
headpunct={},
postheadspace=1em,
%qed=▣,
]{maintheorem}

\declaretheorem[%
name=정의,
style=maintheorem,
numberwithin=section, shaded={%bgcolor=MainColorThree!20,
margin=.5em}]{dfn}
% \begin{dfn}[]
% \end{dfn}


\usetheme{Hannover}


\title{Cpp for PS}

\author{EUnS}

\begin{document}

\begin{frame}
  \maketitle
\end{frame}

\section{입출력}

\begin{frame}[fragile]{입출력의 변화}
    \begin{lstlisting}[style = CStyle]
    \include{iostream}
        std::ios_base::sync_with_stdio(false); 
        std::cin.tie(NULL);//속도를 위한 흑마법
        std::cout<<" " << ' ' << a << b << c << std::endl << '\n';
        std::cin >> a >> b >> c 
    \end{lstlisting}
    \url{https://www.acmicpc.net/board/view/22716}

    scanf와 printf쓰는 사람들도 많음
\end{frame}
%scanf와 printf가 낫다고 생각한적도있음.

\section{namespace}

\begin{frame}[fragile]{아마 달고살것..}
    std::를 쓰는 시간이 아깝다.

    \begin{lstlisting}[style = CStyle]
        using namespace std;
    \end{lstlisting}
        
\end{frame}

\section{STL}

\begin{frame}[fragile]
    \frametitle{STL}
    
    \begin{lstlisting}[style = CStyle]
        #include{STL}
        std::container<type> name();
        name.method();
    \end{lstlisting}

    \begin{itemize}
        \item queue
        \item stack
        \item deque(deck)
        \item vector
    \end{itemize}
    공통 메소드
    \begin{itemize}
        \item empty() : 큐가 비어있으면 true 아니면 false를 반환
        \item size() : 큐 사이즈를 반환
    \end{itemize}
\end{frame}


\begin{frame}
    \frametitle{Queue}
    $\#include<queue>$
    \begin{itemize}
        \item push(element) : 큐에 원소를 추가(뒤에)
        \item pop() : 큐에 있는 원소를 삭제(앞에) 조회
        \item front() : 큐 제일 앞에 있는 원소를 반환
        \item back() : 큐 제일 뒤에 있는 원소를 반환
    \end{itemize}
\end{frame}



\subsection{stack}


\begin{frame}
    \frametitle{stack}
    $\#include<stack>$
    \begin{itemize}
        \item push(element) : top에 원소를 추가
        \item pop() : top에 있는 원소를 삭제
        \item top() : top(스택의 처음이 아닌 가장 끝)에 있는 원소를 반환
    \end{itemize}
\end{frame}


\subsection{deque}

\begin{frame}
    \frametitle{deque}

    $\#include<deque>$
    
    stack  + queue
    
    너무 많아서 생략.

    후의 vector와 거의 동일
\end{frame}




\section{vector}

\begin{frame}    
    \frametitle{vector}
    $\#include<vector>$

    메소드가 매우많으니 자세한건 여기와 cppref참고 : \url{https://blockdmask.tistory.com/70}
    
    자주쓰는것만 정리
    
    \begin{itemize}
        \item resize() : 공간할당
        \item push\_back() : 스택과 동일 
        \item pop\_back() : 스택과 동일
        \item capacity() : 할당된 공간크기 리턴 
        \item begin() : 첫번째 원소 iter 반환
        \item end() : 마지막 원소의 다음 iter 반환
    \end{itemize}

\end{frame}


\begin{frame}
    \frametitle{자료구조에 대한 논의}
\end{frame}

%라이브러리 뜯어서 설명한다.

\begin{frame}
    stack은 벡터로 대체 가능하고 결론 벡터 짱짱
    벡터 = 배열이라고 생각할것.
    
\end{frame}
% vector




\section{ect}

\begin{frame}
    \frametitle{그 밖의 자주쓰는것}
    \begin{itemize}
        \item sort
        \item priority queue
        \item abs : 절댓값 반환
        \item tuple
        \item pair
        \item string : $#include <string>$
    \end{itemize}
\end{frame}

\subsection{sort}


\begin{frame}
    \frametitle{Sort}
    msvc에선 introsort사용
    $#include<algorithm>$
    \url{https://www.acmicpc.net/blog/view/22}
\end{frame}

\subsection{priority queue}

\begin{frame}{priority queue 우선순위 큐}
    
    $#include<queue>$
    \url{https://koosaga.com/9}
    
\end{frame}

\begin{frame}{pair}

    \begin{itemize}
        \item $#include<utility>$
        \item 2개 쌍으로 가지는 자료구조 
        \item $std::pair<type,type>$
        \item .first .second
    \end{itemize}
\end{frame}


\begin{frame}{다른 STL들}
    본인도 아직 안써봄
    \begin{itemize}
        \item <array> : 배열
        \item <list> : 양방향 연결리스트
        \item <forward_list> : 단방향 연결리스트
        \item <map> : 이진탐색트리 기반 / 자동정렬 / key - value pair로 구성
        \item <set> : 이진탐색트리 기반 / 자동정렬 / key 만 저장함
        \item <unordered_map> : 정렬되지않은 map
        \item <unordered_set> : 정렬되지 않은 set
    \end{itemize}
    출처: \url{ttps://code-algalon.tistory.com/188}
\end{frame}

\begin{frame}{Q&A}
    질문?
\end{frame}

% \begin{frame}
    
% \end{frame}

\end{document}

%
