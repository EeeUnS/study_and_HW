\documentclass{oblivoir}
    \usepackage{hyperref}
\begin{document}
\title{OOP}
\author{이윤승 201712052}
\maketitle
\tableofcontents

\section{procedural programming}
절차적 프로그래밍은 초기의 프로그래밍 방식으로서 함수화로 이루어진다. 하지만 시간이 지날수록 만드는 프로그램의 규모가 커짐에 따라 지나치게 길어지는 코드길이와 기존 함수의 이름이 겹치는 문제가 생기는 등 개발, 유지보수가 매우 힘들어져 객체지향 프로그래밍 방식과 함께 언어들이 개발되기 시작했다.

\section{Object-Oriented Programming}
\textbf{객체지향프로그래밍}은 기존의 함수형태에서 더큰 클래스,모듈 단위로서 코드들을 함수보다 더 큰덩이로 묶고 파일 분할을 함으로 덩이째로 잘라서 완전히 독립된 상태로 볼 수 있게한다. 객체지향프로그래밍은 데이터들과 그 데이터들을 관리, 조작하는 함수(method)로 구성되어 이것들을 하나의 독립된 개체로서 바라보는 것이다. 이를함으로서 실제 개발단계에서의 분업화와 데이터관리에 대한 다양한 기능들을 지원하여 유지보수 또한 편하게 할수있게 한다. 


\section{캡슐화} 
여기서 꼭 중요한 객체지향프로그래밍 설계의 기본은 데이터들을 꼭 프라이빗으로 설정하여 내부 메소드로만 접근을 하게하고 외부의 접근을 막는것이다. 이렇게 함으로써 데이터들의 변경되는 부분을 찾아서 유지보수를 쉽게하게되었다.


\section {상속} 
클래스의 확장이라고도 볼수있는데 하나의 객체를 만들었는데 그 객체를 완전히 포함하는 더 큰 객체가 필요할때 기존의 객체를 이용해서 추가적인 부분만 기술하게 한다.

업캐스팅은 부모클래스의 자료형이 자식클래스를 가르키는경우이다.다운캐스팅은 업캐스팅된 부모클래스의 자료형을 다시 자식클래스의 자료형으로 바꿔주는것이다.

\section {다형성}
기존 동일한 함수이름이 사용이 불가능한것을 개선해서 인자형에 따라 같은이름의 함수를 맞춰서 제공한다.  다양한 데이터형처리와 함께 템플릿 일반화 프로그래밍을 이끌었다.


\section{해싱,해시함수}
key를 통해서 값을 찾는 방식 일반적으로 탐색이 O(1)이나 최악의 경우O(n)


\section{set}
정렬되지않은 집합 , 중복 원소를 가질수없다 , 해당 원소가 들어있는지를 검색하기위한 bool 함수를 사용한다.
multiset은 하나의 원소를 여러번 가질수있다

\section{레드블랙트리}
특이한 이진 트리인데 각노드에 레드나 블랙 색이 있으며 다음의 특징이 있다. 
\begin{itemize}
    \item 노드는 레드 혹은 블랙 중의 하나이다.
    \item 루트 노드는 블랙이다.
    \item 모든 널 포인터는 블랙이다.
    \item 레드 노드의 자식노드 양쪽은 언제나 모두 블랙이다. (즉, 레드 노드는 연달아 나타날 수 없으며, 블랙 노드만이 레드 노드의 부모 노드가 될 수 있다)
    \item 어떤 노드로부터 시작되어 리프 노드에 도달하는 모든 경로에는 리프 노드를 제외하면 모두 같은 개수의 블랙 노드가 있다.
\end{itemize}

\section{BST(Binary Search Tree) 중위순회결과}
\begin{itemize}
    \item 이진탐색트리 : 각 노드에 값이 있고 왼쪽 서브트리는 해당 노드의 값보다 작으며 오른쪽은 같거나 큰값으로 이루어져있다.
    \item  중위순회 : 
    \begin{enumerate}
        \item 왼편 서브트리(left subtree)를 중위 순회한다.
        \item 루트 노드(root node)를 방문한다.
        \item 오른편 서브트리(right subtree)를 중위 순회한다.
    \end{enumerate}
\end{itemize}
이진탐색트리를 중위순회했을때 값을 오름차순으로 순회하게된다.

\url{https://kingpodo.tistory.com/28}
Thinking
    
\end{document}