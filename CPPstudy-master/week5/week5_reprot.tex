\documentclass{oblivoir}
    \usepackage{hyperref}

\begin{document}
\title{simple report}
\author{이윤승 201712052}
\maketitle
\tableofcontents
\section{Linear Date Structure}
덱빼곤 아는거니 빠르게 생략
\begin{itemize}
    \item  리스트(list): 양방향, 단방향, 
    \item  스택(stack):LIPO 
    \item  큐(queue): PIPO
    \item  덱(deck, deque) : 큐의 형태에서 앞 뒤  push,pop이 둘다 가능
\end{itemize} 

\section{Nonlinear Date Structure}
일자로 이루어진 선형 자료구조와는 다르게 일반적인 경우이다.
\begin{itemize}
    \item 그래프 : 정점과 간선으로 이루어져있으며 간선에 가중치가 있을수도있고 없을 수도있고 방향이 간선 방향이 있을수도있고 없을수도 있음 생략
    \item  트리 : 고3때 확통 공부를 하다보면 써보는 수형도이다. 엄연히 그래프의 일종이다. 
\end{itemize}
 

배열 : 생략

연결리스트 : 생략

이진 탐색 트리 : 하나의 정점에 연결된 간선이 두개인 트리

해시테이블 : 어려워서 생략

\section{STL(Standard Template Library)}

\begin{itemize}
    \item vector : 동적 배열 구현, 배열과같이 메모리가 붙어서 존재한다. 인덱스접근가능
    \item list : 노드기반 양방향 연결리스트 구현 무조건 순환을해서 탐색을 해야함
    \item stack : 스택 구현
    \item queue : 큐 구현
    \item deque : 덱 구현
    \item map : 트리,정렬 키와 데이터 같이 저장 키값 중복 삽입 불가능 , 빠른 탐색이 장점, 삭제가 불편함. 
    구현 그림 참고 : \url{https://blockdmask.tistory.com/87}
    \item set : 집합, 정렬X 중복데이터 삽입 불가능
\end{itemize}

\section{c++11에 추가된 표준 라이브러리}
\begin{itemize}
    \item array : array를 아예 템플릿 클래스로 만들어서 갖가지 기능을 넣어놓음
    \item unordered\_map : 정렬되지않은 map
    \item foward\_list : 단방향 리스트
    \item unordered\_set : 정렬되지않은 set
\end{itemize}


\end{document}